**styles.css**
En el CSS le aplicamos estos estilos CSS aplican diseño y formato al contenido de la página web, 
definiendo cómo se ven y se comportan los elementos como productos, imágenes, detalles de laptop 
y el pie de página.


*Estilos Generales (body):*
font-family: Establece que el texto dentro del cuerpo del documento se mostrará en la 
fuente Arial o cualquier fuente sans-serif disponible si Arial no está disponible.

margin y padding: Ambos establecidos en 0 para asegurarse de que no haya espacio extra
alrededor del borde exterior del cuerpo.
background-color: Establece el color de fondo del cuerpo en un tono verde claro (#04fd0c).

*Estilos del Encabezado (header):*
background-color: Define el color de fondo del encabezado como un púrpura oscuro (#6718c0).

color: Define el color del texto del encabezado como un verde claro (#9af630).

padding: Agrega un relleno de 10px arriba y abajo del encabezado, pero ningún 
relleno a los lados.

text-align: Centra el contenido del encabezado horizontalmente en la página.


*Estilos para la barra de navegación (nav ul y sus elementos hijos):*
"nav ul"
list-style: none;: Elimina los puntos de lista predeterminados en los elementos
de lista (li).
padding: 0;: Elimina el relleno predeterminado que los navegadores añaden a las
listas no ordenadas (ul).

"nav ul li"
display: inline;: Hace que los elementos de lista (li) se muestren en línea uno
al lado del otro en lugar de en bloques separados.
margin: 0 15px;: Establece márgenes de 0 píxeles arriba y abajo, y 15 píxeles a la
izquierda y derecha de cada elemento de lista. Esto crea espacios uniformes entre
los elementos de la lista en la barra de navegación.

"nav ul li a"
color: #51ff00;: Define el color del texto de los enlaces (a) dentro de los 
elementos de lista (li) como verde claro (#51ff00).
text-decoration: none;: Elimina el subrayado predeterminado de los enlaces para
que no aparezcan subrayados cuando se muestran en la página.


*Estilos para el contenido principal (main y section):*
"main"
padding: 20px;: Añade un relleno interior de 20 píxeles alrededor del contenido dentro 
del elemento main. Esto crea un espacio entre el borde del elemento y su contenido.

"section"
margin-bottom: 20px;: Añade un margen inferior de 20 píxeles a cada elemento section. 
Esto separa visualmente las secciones entre sí en la página.

*Estilos para .producto:*
border: 1px solid #6f30c8;: Establece un borde sólido de 1 píxel de grosor con color
#6f30c8 (un tono de púrpura oscuro) alrededor del contenedor .producto.
padding: 10px;: Define un espacio de relleno de 10 píxeles dentro del borde del contenedor .producto.
margin-bottom: 10px;: Establece un margen inferior de 10 píxeles para separar este elemento
del siguiente elemento en la página.
background-color: #ffffff;: Fija el color de fondo del contenedor .producto a blanco (#ffffff).


*Estilos para .producto img:*
max-width: 100°/o;: Asegura que la imagen no se extienda más allá de su contenedor .producto.

width: 165px;: Establece un ancho fijo de 165 píxeles para la imagen.

height: auto;: Permite que la altura de la imagen se ajuste automáticamente 
en proporción al ancho definido.

display: block;: Hace que la imagen se comporte como un bloque, ocupando todo el ancho 
disponible dentro del contenedor .producto.

float: left;: Alinea la imagen a la izquierda dentro de su contenedor, permitiendo que el texto 
flote alrededor de ella.

margin-right: 10px;: Agrega un margen de 10 píxeles a la derecha de la imagen para separarla 
del texto o de otros elementos adyacentes.


*Estilos para .detalle-laptop:*
background-color: #ffffffec;: Define un color de fondo ligeramente transparente 
(#ffffffec equivale a un blanco con una opacidad del 92°/o) para el contenedor .detalle-laptop.

padding: 10px;: Establece un espacio de relleno de 10 píxeles dentro del borde del contenedor .detalle-laptop.

margin-bottom: 100px;: Agrega un margen inferior de 100 píxeles para separar este elemento 
del siguiente elemento en la página.

border: 1px solid #9900ff;: Aplica un borde sólido de 1 píxel de grosor con color 
#9900ff (un tono de púrpura brillante) alrededor del contenedor .detalle-laptop.


*Estilos para footer:*
background-color: #622fda;: Fija el color de fondo del pie de página (footer) a #622fda (un tono de morado).

color: #000000;: Establece el color del texto dentro del pie de página a negro (#000000).

text-align: center;: Centra el contenido del pie de página horizontalmente.

padding: 10px 0;: Añade un relleno de 10 píxeles en la parte superior e inferior del pie de página 
y ningún relleno en los lados.

position: fixed;: Fija el pie de página en una posición fija en la parte inferior de 
la ventana del navegador.

width: 100°/o;: Asegura que el pie de página ocupe todo el ancho disponible del navegador.

bottom: 0;: Alinea el pie de página en la parte inferior de la ventana del navegador.