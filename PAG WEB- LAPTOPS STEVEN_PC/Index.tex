**Index.html**
El HTML ayuda a estructurar mi tienda en línea para laptops "STEVEN PC" con
secciones bien definidas para navegar como la bienvenida, la descripción,
ver productos, añadir productos al carrito y contactar a la tienda.

-<!DOCTYPE html>: Esto indica al navegador que el documento es un HTML5,
el estándar más reciente de HTML.

-<html lang="es">: Define que el idioma principal del documento es español (es).

-<head>: Aquí empieza la sección de metadatos y enlaces a recursos externos.

-<meta charset="UTF-8">: Especifica que el documento está codificado en UTF-8, 
que incluye caracteres especiales como letras acentuadas.

<meta name="viewport" content="width=device-width, initial-scale=1.0">: Ayuda a 
que el diseño del sitio web se adapte correctamente a diferentes tamaños de
pantalla, como los de dispositivos móviles.

<title>Tienda de Laptops</title>: Define el título de la página 
que aparecerá en la pestaña del navegador.

<link rel="stylesheet" href="styles.css">: Enlaza la hoja de estilos
externa styles.css para aplicar estilos al contenido HTML.

<body>: Inicia el cuerpo del documento HTML donde se encuentra 
el contenido visible de la página.

<header>: Define la cabecera de la página, que generalmente contiene
el título principal y la navegación.

<h1>Tienda de Laptops "STEVEN PC"</h1>: Es el título principal de la página,
mostrado como un encabezado de nivel 1.

<nav>: Define la sección de navegación de la página.

<ul>: Crea una lista desordenada de elementos.

<li><a href="#inicio">Inicio</a></li>, <li><a href="#productos">Productos</a></li>, 
<li><a href="#contacto">Contacto</a></li>: Son elementos de la lista que contienen 
enlaces (<a>) a diferentes secciones de la página, usando IDs 
(#inicio, #productos, #contacto) como destinos.

<main>: Define el contenido principal de la página, excluyendo la cabecera
y el pie de página.

<section id="inicio">: Crea una sección con el ID inicio, que se puede usar 
para enlazar desde la navegación.

<h2>Bienvenido</h2>: Encabezado de nivel 2 que dice "Bienvenido".

<p>Encuentra las mejores laptops al mejor precio, solo en STEVEN PC!!</p>: Párrafo 
que describe el propósito de la tienda.

<section id="productos">: Crea una sección para mostrar los productos disponibles.

<h2>Nuestros Productos</h2>: Encabezado de nivel 2 que describe la sección de productos.

<div class="producto">: Contenedor para cada producto listado.

<h3>LAPTOP 15 LENOVO AMD RYZEN 3</h3>: Nombre del producto como encabezado de nivel 3.

<a href="LaptopRyzen3.html">: Enlace a una página específica para este producto.

<img src="images/LaptopRyzen3.jpg" alt="LAPTOP 15 LENOVO AMD RYZEN 3">: Imagen del 
producto con una descripción alternativa.

<p>LAPTOP 15" - LENOVO AMD Ryzen 3 3250U</p>: Descripción corta del producto.
<p>16 GB DDR4 RAM, 1T SSD, cámara web,</p>, <p>HDMI, WiFi 5, peso ligero gris platino,
 Windows 10</p>: Detalles técnicos del producto.

<p>Precio: 399.99</p>: Precio del producto.

<section id="contacto">: Sección para permitir a los usuarios contactar a la tienda.

<h2>Contacto</h2>: Encabezado de nivel 2 que indica que es la sección de contacto.

<form id="contacto-form">: Formulario para enviar mensajes.

<label for="nombre">Nombre:</label>: Etiqueta para el campo de nombre.

<input type="text" id="nombre" name="nombre">: Campo de entrada donde el usuario 
puede escribir su nombre.

<label for="email">Email:</label>: Etiqueta para el campo de correo electrónico.

<input type="email" id="email" name="email">: Campo de entrada donde el usuario 
puede escribir su correo electrónico.

<label for="mensaje">Mensaje:</label>: Etiqueta para el campo de mensaje.

<textarea id="mensaje" name="mensaje"></textarea>: Área de texto donde el usuario
puede escribir su mensaje.

<button type="submit">Enviar</button>: Botón para enviar el formulario.

<footer>: Define el pie de página de la página web.

<p>&copy; 2024 Tienda de Laptops</p>: Texto que muestra el año actual y
el nombre de la tienda, seguido del símbolo de derechos de autor.

<script src="scripts.js"></script>: Enlaza un archivo JavaScript 
externo (scripts.js) para agregar interactividad y funcionalidad a la página.


